\section*{От переводчика}

Я рад, что вы открыли этот перевод. Если вы прочитаете хотя бы одну главу, я уже буду считать, что мои усилия потрачены не зря.

На момент, когда я начал этот перевод, данная книга помогла мне заполнить множество пробелов в моих собственных знаниях о библиотеке React и успешно пройти интервью на соответствующую позицию. 

Этот перевод предназначен прежде всего для тех, кто хочет изучить различные аспекты использования React, а чтение оригинала вызывает затруднения. При прочих равных, конечно рекомендую обращаться к первоисточнику.

Эта книга скорее всего не для вас, если у вас за спиной многие годы разработки web приложений в целом и React приложений в частности, хотя и в данном случае могут пригодиться отдельные главы. 

На момент окончания перевода в мире React уже во всю правят балом хуки, и "внешний вид" кода несколько изменился, что стоит учитывать при прочтении данной книги. Однако, множество вещей, таких как стили и тестирование, не потеряли своей актуальности.

Все вопросы и замечания по этому переводу можно присылать через \href{https://github.com/obabichev/React-Design-Patterns-and-Best-Practices-Rus}{GitHub (Link)} или по почте (babichev.oleg.n@gmail.com).

Еще раз спасибо, что открыли данный перевод, успехов в освоении его содержимого.

\begin{flushright}
Бабичев Олег \\
Октябрь 2019 г.
\end{flushright}
